% !TeX spellcheck = it_IT
% --------------------------------------------------------------------
% Preamble
% --------------------------------------------------------------------
\documentclass[paper=a4, fontsize=11pt,twoside]{scrartcl}   % KOMA

\usepackage[a4paper,pdftex]{geometry}   % A4paper margins
\usepackage[utf8]{inputenc}
\setlength{\oddsidemargin}{5mm}         % Remove 'twosided' indentation
\setlength{\evensidemargin}{5mm}

\usepackage[english]{babel}
\usepackage[protrusion=true,expansion=true]{microtype}  
\usepackage{amsmath,amsfonts,amsthm,amssymb}
\usepackage{graphicx}

% --------------------------------------------------------------------
% Definitions (do not change this)
% --------------------------------------------------------------------
\newcommand{\HRule}[1]{\rule{\linewidth}{#1}}   % Horizontal rule

\makeatletter                           % Title
\def\printtitle{%                       
	{\centering \@title\par}}
\makeatother                                    

\makeatletter                           % Author
\def\printauthor{%                  
	{\centering \large \@author}}               
\makeatother                            

% --------------------------------------------------------------------
% Metadata (Change this)
% --------------------------------------------------------------------
\title{ \normalsize \textsc{Relazione Algoritmi e Strutture Dati}    % Subtitle
	\\[2.0cm]                               % 2cm spacing
	\HRule{0.5pt} \\                        % Upper rule
	\LARGE \textbf{\uppercase{Implementazione dizionario in linguaggio 
			C con Red-Black Tree}}    % Title
	\HRule{2pt} \\ [0.5cm]      % Lower rule + 0.5cm spacing
	\normalsize 31 Maggio 2017          % Todays date
}

\author{
	Nicolas Farabegoli\\
	Paolo Baldini\\
	Università di Bologna\\  
	\texttt{nicolas.farabegoli@studio.unibo.it} \\
	\texttt{paolo.baldini@studio.unibo.it}\\
}

\begin{document}
	% ------------------------------------------------------------------------------
	% Maketitle
	% ------------------------------------------------------------------------------
	\thispagestyle{empty}       % Remove page numbering on this page
	
	\printtitle                 % Print the title data as defined above
	\vfill
	\printauthor                % Print the author data as defined above
	\newpage
	
	% ------------------------------------------------------------------------------
	% Begin document
	% ------------------------------------------------------------------------------
	\setcounter{page}{1}        % Set page numbering to begin on this page
	
	
	\section{Analisi}
		L'obiettivo del progetto è quello di costruire e gestire un vocabolario 
		contenente le voci, ordinate, e le corrispondenti definizioni. 
		Il dizionario può essere inizialmente generato a partire da un file di testo.  
		In questo caso si vogliono memorizzare tutte le parole presenti in un file di 
		testo in ordine lessico grafico.
		
		Si utilizzi per rappresentare il dizionario la struttura dati che si ritiene 
		più opportuna, dopo una attenta analisi. Si possono utilizzare le strutture 
		dati astratte affrontate dal corso (alberi, hash table, code, pile, liste, 
		grafi, ... ).
		
		Il file contiene un testo le cui parole non sono memorizzate in ordine lessicografico, e sono separate da spazio. I vocaboli sono inizialmente salvati senza la corrispondente definizione; la definizione, una stringa di dimensione sconosciuta, sarà inserita in un secondo momento, tramite la funzione apposita.
		
		Ogni voce del dizionario contiene al massimo 20 caratteri, e minimo 2, e le definizioni inserite ne contengono al massimo 50.
		
	\subsection{Vincoli}
		Nel dizionario non devono essere inserite le parole di un solo carattere o con più di 10 caratteri. Inoltre non devono essere inserite parole che differiscono solo per caratteri minuscoli/miuscoli. Le parole e le definizioni inserite nel dizionario non devono presentare caratteri maiuscoli; nel caso di inserimento di un parola con un carattere, o più, maiuscoli, convertirli per inserirli nel modo corretto. 
		
		Si assume che nel file di testo e nel dizionario non ono presenti nomi proprio (di persona) e le voci nel dizionario possono ammettere caratteri accentati (è, à, etc...), ma non ammettono cifre e caratteri di puneggiatura.
		
	\subsection{Funzionalità richieste}
		Sono presenti metodi di gestione del dizionario come il conteggio delle voci del dizionario, l'inserimento o la cancellazione di un nuovo vocabolo, ...
		
		Esistono due diverse funzioni di ricerca:
		\begin{itemize}
			\item la prima, di base, dato un termine restituisce, se presente, la sua definizione
			\item la seconda è una funzione di searchAdvance che verifica la presenza della parola da cercare e restituisce la tre parole più simile a quella che si vuole cercare. Per similarità si intende il minor numero di modifiche (sostituzioni, inserimenti) per trasformare una parola nell'altra.
		\end{itemize}			
		 Le specifiche della ricerca avanzata, e l'algoritmo da utilizzare, sono a scelta dello studente. Inoltre si vuole avere la possibilità di caricare e salvare il dizionario, a partire dalla struttura dati ottenuta, con il seguente formato, in un file testuale (\textbf{.txt}):
		 
\end{document}